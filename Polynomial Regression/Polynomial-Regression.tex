% Options for packages loaded elsewhere
\PassOptionsToPackage{unicode}{hyperref}
\PassOptionsToPackage{hyphens}{url}
%
\documentclass[
]{article}
\usepackage{amsmath,amssymb}
\usepackage{lmodern}
\usepackage{iftex}
\ifPDFTeX
  \usepackage[T1]{fontenc}
  \usepackage[utf8]{inputenc}
  \usepackage{textcomp} % provide euro and other symbols
\else % if luatex or xetex
  \usepackage{unicode-math}
  \defaultfontfeatures{Scale=MatchLowercase}
  \defaultfontfeatures[\rmfamily]{Ligatures=TeX,Scale=1}
\fi
% Use upquote if available, for straight quotes in verbatim environments
\IfFileExists{upquote.sty}{\usepackage{upquote}}{}
\IfFileExists{microtype.sty}{% use microtype if available
  \usepackage[]{microtype}
  \UseMicrotypeSet[protrusion]{basicmath} % disable protrusion for tt fonts
}{}
\makeatletter
\@ifundefined{KOMAClassName}{% if non-KOMA class
  \IfFileExists{parskip.sty}{%
    \usepackage{parskip}
  }{% else
    \setlength{\parindent}{0pt}
    \setlength{\parskip}{6pt plus 2pt minus 1pt}}
}{% if KOMA class
  \KOMAoptions{parskip=half}}
\makeatother
\usepackage{xcolor}
\usepackage[margin =2cm]{geometry}
\usepackage{listings}
\newcommand{\passthrough}[1]{#1}
\lstset{defaultdialect=[5.3]Lua}
\lstset{defaultdialect=[x86masm]Assembler}
\usepackage{graphicx}
\makeatletter
\def\maxwidth{\ifdim\Gin@nat@width>\linewidth\linewidth\else\Gin@nat@width\fi}
\def\maxheight{\ifdim\Gin@nat@height>\textheight\textheight\else\Gin@nat@height\fi}
\makeatother
% Scale images if necessary, so that they will not overflow the page
% margins by default, and it is still possible to overwrite the defaults
% using explicit options in \includegraphics[width, height, ...]{}
\setkeys{Gin}{width=\maxwidth,height=\maxheight,keepaspectratio}
% Set default figure placement to htbp
\makeatletter
\def\fps@figure{htbp}
\makeatother
\setlength{\emergencystretch}{3em} % prevent overfull lines
\providecommand{\tightlist}{%
  \setlength{\itemsep}{0pt}\setlength{\parskip}{0pt}}
\setcounter{secnumdepth}{-\maxdimen} % remove section numbering
\usepackage{titlesec}
\titleformat{\paragraph}
   {\normalfont\bfseries}
   {}
   {0pt}
   {}
\lstset{
  breaklines=TRUE
}

\usepackage[T1]{fontenc}
\usepackage{babel}
\usepackage{geometry}
\usepackage{titling}
\usepackage{blindtext}

\setlength{\droptitle}{-4em}     % Eliminate the default vertical space
\addtolength{\droptitle}{-4pt}   % Only a guess. Use this for adjustment
\pretitle{\begin{center} \vspace{10cm}}
\ifLuaTeX
  \usepackage{selnolig}  % disable illegal ligatures
\fi
\IfFileExists{bookmark.sty}{\usepackage{bookmark}}{\usepackage{hyperref}}
\IfFileExists{xurl.sty}{\usepackage{xurl}}{} % add URL line breaks if available
\urlstyle{same} % disable monospaced font for URLs
\hypersetup{
  pdftitle={Linear Regression Analysis},
  pdfauthor={B.M Njuguna},
  hidelinks,
  pdfcreator={LaTeX via pandoc}}

\title{\textbf{Linear Regression Analysis}}
\author{\textbf{B.M Njuguna}}
\date{\textbf{2022-09-02}}

\begin{document}
\maketitle

\newpage
\tableofcontents
\newpage

\hypertarget{introduction}{%
\section{1.0 Introduction}\label{introduction}}

In some cases, the relationship between the response variable \(y\) and
the independent or predictor variable or variables might not be linear.
In a such a case, we cannot apply the linear regression analysis as the
assumption of linearity is violated. Thus, the \textbf{Polynomial
Regression} is a type of regression whereby the relationship between the
response variable and the predictor variables is modeled as the
\(n^{th}\) degree polynomial. The polynomial regression fits a
non-linear relationship between the values of the independent variable
\(X\) and the conditional mean of \(y\) which is denoted as
\(\mathbf{E}(y|x)\). Although the polynomial regression fits a nonlinear
model to the data, as a statistical estimation problem it is linear in
the sense that the conditional mean of \(y\) i.e \(\mathbf{E}(y|x)\) is
linear to the unknown to the parameters estimated from the data, hence
it is referred to as a special case of multilple linear regression.

With polynomial regression, data is approximated using a polynomial
equation of degree \(n\) written as;

\[\mathbf{f(x)}=\alpha_0+\alpha_1x+\alpha_2x_1+\dots+\alpha_nx^n\]

where \(\alpha\) is the set of coefficients.

Now the polynomial regression equation can be written as;

\[y=\beta_0+\beta_1x_i+\beta_2x_i^2+\dots+\beta_nx_i^m+\epsilon_i \\for\space i=1,2,3,\dots,n\]

The above model can be expressed matrix form in terms of a design matrix
\(\mathbb{x}\), response vector \(\vec{y}\), the parameter vector
\(\vec{\beta}\) and the random vector \(\vec{\epsilon }\), as follows;

\[\begin{bmatrix}y_1\\y_2\\y_3\\\vdots\\y_n \end{bmatrix}=\begin{bmatrix}1&x_1&x_2^2&\dots&x_n^m\\1&x_2&x_2^2&\dots&x_2^m\\1&x_3&x_3^2&\dots&x_3^m\\\vdots&\vdots&\vdots&\ddots&\vdots\\1&x_n&x_n^2&\dots&x_n^m\end{bmatrix}\begin{bmatrix}\beta_0\\\beta_1\\\beta_3\\\vdots\\\beta_m\end{bmatrix}\begin{bmatrix}\epsilon_1\\\epsilon_2\\\epsilon_3\\\vdots\\\epsilon_n\end{bmatrix}\]

Which can be written as;

\[\vec{y}=\mathbf{x}\vec{\beta}+\vec{\epsilon}\]

The polynomial coefficients \(\beta\) can be estimated using the
\textbf{ordinary Least Square} as follows;

\[\vec{\beta}=(\mathbf{X}^T\mathbf{X})^{-1}\mathbf{X}\vec{y}\]

It is often difficult to interpret individual polynomial regression
coefficients since the underlying monomials are highly correlated for
example if \(x\) is \textbf{Uniformly distributed},then \(x\) and
\(x^2\) have a high correlation of 0.97. Therefore, it is generally
informative to consider the fitted regression function as a
whole.\footnote{monimal \emph{in plural monomials} is an algebraic
  expression consisting of one term.}

In r, we use the function \textbf{poly()} which is in the basic syntax
to fit a polynomial regression model to data.

\end{document}
